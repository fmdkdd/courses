\documentclass[a4paper, 12pt]{article}
\addtolength{\topmargin}{-.8in}
\addtolength{\textheight}{1.6in}

\usepackage{lastpage}
\usepackage{fancyhdr}
\fancypagestyle{plain}{%
  \fancyfoot[C]{\thepage{}/\pageref{LastPage}}
}
\renewcommand{\headrulewidth}{0pt}
\pagestyle{plain}

\usepackage[bitstream-charter]{mathdesign}
\usepackage{fontspec}
%% \setmainfont{Charter}
\setmonofont[Scale=MatchLowercase]{Fira Mono}

\usepackage{wasysym}
\newcommand{\choice}[1]{\Square\hspace{2pt} #1\hspace{5pt}}
\newcommand{\choicec}[1]{\Square\hspace{2pt} \lstinline{#1}\hspace{5pt}}

\usepackage{listings}
\lstset{basicstyle=\ttfamily}
\lstset{tabsize=3, columns=fullflexible, keepspaces=true}
\lstset{breaklines=false, showstringspaces=false}
\lstset{frame=leftline, framerule=1pt}
\lstset{framesep=2pt}
\lstdefinelanguage{JavaScript}{
keywords={new, true, false, try, catch, var, let, function, return, null, switch, if, in, of, while, do, else, case, break},
keywordstyle=\bfseries,
ndkeywords={class, export, boolean, throw, implements, import, this},
ndkeywordstyle=\bfseries,
identifierstyle=,
sensitive=false,
comment=[l]{//},
morecomment=[s]{/*}{*/},
morestring=[b]',
morestring=[b]"
}

\providecommand{\alert}[1]{\textbf{#1}}

\title{Quiz JavaScript}
\author{FIL A2}
\date{16 janvier 2018}

\begin{document}

\maketitle

\textbf{Nom} :

Sans machine ni document.

Durée : 20 minutes.

Cocher la bonne réponse (il peut y en avoir plusieurs).

\begin{enumerate}
\item \lstset{language=javascript}
\begin{lstlisting}
 var a = 2;
\end{lstlisting}

  Que vaut \lstinline{a} ?

  \choicec{"2"} \choicec{2} \choicec{a} \choicec{undefined}
\item \lstset{language=javascript}
\begin{lstlisting}
 var a;
 var a = 2;
\end{lstlisting}

  Que vaut \lstinline{a} ?

  \choicec{"2"} \choicec{2} \choicec{a} \choicec{undefined} \choice{C'est une erreur}
\item \lstset{language=javascript}
\begin{lstlisting}
 a = 2;
 let a;
\end{lstlisting}

  Que vaut \lstinline{a} ?

  \choicec{"2"} \choicec{2} \choicec{a} \choicec{undefined} \choice{C'est une erreur}
\item \lstset{language=javascript}
\begin{lstlisting}
 let a = 2;
 let a;
\end{lstlisting}

  Que vaut \lstinline{a} ?

  \choicec{"2"} \choicec{2} \choicec{a} \choicec{undefined} \choice{C'est une erreur}
\item \lstset{language=javascript}
\begin{lstlisting}
 a = 1 / 2;
\end{lstlisting}

  Que vaut \lstinline{a} ?

  \choicec{0.5} \choicec{1/2} \choicec{"1/2"} \choicec{1}
\item \lstset{language=javascript}
\begin{lstlisting}
 1 == '1';
\end{lstlisting}

  Que vaut cette expression ?

  \choicec{true} \choicec{false} \choicec{1} \choicec{undefined}
  \newpage
\item \lstset{language=javascript}
\begin{lstlisting}
 1 === '1';
\end{lstlisting}

  Que vaut cette expression ?

  \choicec{true} \choicec{false} \choicec{1} \choicec{undefined} \choice{C'est une erreur}
\item \lstset{language=javascript}
\begin{lstlisting}
 null === undefined;
\end{lstlisting}

  Que vaut cette expression ?

  \choicec{true} \choicec{false} \choicec{null} \choicec{undefined}
\item \lstset{language=javascript}
\begin{lstlisting}
 'abc' === ('ab' + 'c');
\end{lstlisting}

  Que vaut cette expression ?

  \choicec{true} \choicec{false} \choicec{'abc'} \choicec{undefined}
\begin{lstlisting}
 let f = function(x,y) {
   return y;
 }
\end{lstlisting}

  Que vaut \lstinline{f(1,2)} ?

  \choicec{1} \choicec{2} \choicec{undefined}

  Que vaut \lstinline{f(1)} ?

  \choicec{1} \choicec{2} \choicec{undefined} \choice{C'est une erreur}

  Que vaut \lstinline{f(1,2,3)} ?

  \choicec{1} \choicec{2} \choicec{3} \choicec{undefined} \choice{C'est une erreur}
\item \lstset{language=javascript}
\begin{lstlisting}
 function f(x) {
   let o = [x];
   return function(y) {
     o.push(y);
     return o.length;
   };
 }
 let g = f('b');
 g('c');
 g('d');
\end{lstlisting}

  Que retourne \lstinline{f('a')} ?

  \choicec{0} \choicec{1} \choicec{2} \choicec{3} \choice{une fonction} \choicec{undefined}

  Que retourne \lstinline{f('a')('a')} ?

  \choicec{0} \choicec{1} \choicec{2} \choicec{3} \choice{une fonction} \choicec{undefined}

  Que retourne l'appel \lstinline{g('c')} ?

  \choicec{0} \choicec{1} \choicec{2} \choicec{3} \choice{une fonction} \choicec{undefined}

  Que retourne l'appel \lstinline{g('d')} ?

  \choicec{0} \choicec{1} \choicec{2} \choicec{3} \choice{une fonction} \choicec{undefined}
\item \lstset{language=javascript}
\begin{lstlisting}
let obj = {
  a: "b",
  b: 2
};
\end{lstlisting}

  Que vaut \lstinline{obj.a} ?

  \choicec{2} \choicec{"b"} \choicec{obj} \choicec{undefined}

  Que vaut \lstinline{obj["b"]} ?

  \choicec{2} \choicec{"b"} \choicec{obj} \choicec{undefined}

  Que vaut \lstinline{obj[obj.a]} ?

  \choicec{2} \choicec{"b"} \choicec{obj} \choicec{undefined}

  Que vaut \lstinline{obj[2]} ?

  \choicec{2} \choicec{"b"} \choicec{obj} \choicec{undefined}
\item \begin{lstlisting}
let P = {
  x: 0,
  y: 1,
};
let O = {__proto__: P,
  x: 2,
};
O.y = 3;
\end{lstlisting}

  Que vaut \lstinline{O.x} ?

  \choicec{0} \choicec{1} \choicec{2} \choicec{3} \choicec{undefined}

  Que vaut \lstinline{O.y} \emph{avant} l'affectation \lstinline{O.y = 3} ?

  \choicec{0} \choicec{1} \choicec{2} \choicec{3} \choicec{undefined}

  Que vaut \lstinline{O.y} \emph{après} l'affectation \lstinline{O.y = 3} ?

  \choicec{0} \choicec{1} \choicec{2} \choicec{3} \choicec{undefined}

  Que vaut \lstinline{P.y} \emph{après} l'affectation \lstinline{O.y = 3} ?

  \choicec{0} \choicec{1} \choicec{2} \choicec{3} \choicec{undefined}

\newpage
\item Écrivez la fonction \lstinline{intersection} qui retourne un nouveau
  tableau qui contient les éléments communs aux deux tableaux passés en
  paramètres.  Utilisez l'égalité stricte pour comparer les éléments.

  \framebox(400,480){}
\end{enumerate}

\end{document}
