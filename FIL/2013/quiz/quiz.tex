% Created 2012-12-11 Tue 07:56
\documentclass[a4paper, 12pt]{article}
\usepackage{fontspec}
\usepackage{luatextra}
\usepackage{hyperref}
\usepackage{graphicx}
\addtolength{\topmargin}{-.8in}
\addtolength{\textheight}{1.6in}
\setmainfont[Mapping=tex-text]{Liberation Serif}
\newfontfamily\progfont[Scale=MatchLowercase]{Ubuntu Mono}
\usepackage{wasysym}
\newcommand{\choice}[1]{\Square\hspace{2pt} #1\hspace{5pt}}
\newcommand{\choicec}[1]{\Square\hspace{2pt} \lstinline{#1}\hspace{5pt}}
\usepackage{listings}
\lstset{basicstyle=\progfont}
\lstset{tabsize=3, columns=fullflexible, keepspaces=true}
\lstset{breaklines=false, showstringspaces=false}
\lstset{frame=leftline, framerule=1pt}
\lstset{framesep=2pt}
\lstdefinelanguage{JavaScript}{
keywords={typeof, new, true, false, catch, function, return, null, catch, switch, var, if, in, while, do, else, case, break},
keywordstyle=\bfseries,
ndkeywords={class, export, boolean, throw, implements, import, this},
ndkeywordstyle=\bfseries,
identifierstyle=,
sensitive=false,
comment=[l]{//},
morecomment=[s]{/*}{*/},
morestring=[b]',
morestring=[b]"
}
\providecommand{\alert}[1]{\textbf{#1}}

\title{Quiz JavaScript}
\author{Programmer le web avec JavaScript A2/FIL}
\date{7 février 2014}

\begin{document}

\maketitle

\textbf{Nom} :

Sans machine ni document.

Durée : 10 minutes.

Cocher la bonne réponse (il peut y en avoir plusieurs).

\begin{enumerate}
\item \lstset{language=javascript}
\begin{lstlisting}
var a = 2 + 3;
\end{lstlisting}

  Que vaut \lstinline{a} ?

  \choicec{5} \choicec{"a"} \choicec{undefined} \choice{Donald Duck}
\item \lstset{language=javascript}
\begin{lstlisting}
var f = function (x) {
   return x;
}
\end{lstlisting}

   Que vaut \lstinline{f()} ?

   \choicec{1} \choicec{2} \choicec{undefined}

   Que vaut \lstinline{f(1,2)} ?

   \choicec{1} \choicec{2} \choicec{undefined}
\item \lstset{language=javascript}
\begin{lstlisting}
var x = 5;
function plus(y) {
  return x + y;
}
\end{lstlisting}

  Que vaut \lstinline{plus(3)} ?

  \choicec{3} \choicec{5} \choicec{8} \choicec{undefined}
\item \lstset{language=javascript}
\begin{lstlisting}
var obj = {
  x: 12
};
obj.y = 5 + obj.x;
delete obj.x;
obj.z = obj.x;
obj.x = 6;
\end{lstlisting}

  Que vaut \lstinline{obj.x} ?

  \choicec{12} \choicec{5} \choicec{17} \choicec{6} \choicec{undefined}

  Que vaut \lstinline{obj.y} ?

  \choicec{12} \choicec{5} \choicec{17} \choicec{6} \choicec{undefined}

  Que vaut \lstinline{obj.z} ?

  \choicec{12} \choicec{5} \choicec{17} \choicec{6} \choicec{undefined}
\item \lstset{language=javascript}
\begin{lstlisting}
var pacman = {
  x: 0,
  move: function(dx) {
    this.x += dx;
  }
};
var f = pacman.move;
\end{lstlisting}

  Le mot-clé \lstinline{this} de la fonction \lstinline{move} est lié
  à l'objet \lstinline{pacman} lors de la définition de l'objet.

  \choice{Vrai} \choice{Faux}

  Que vaut \lstinline{pacman.x} après l'appel
  \lstinline{pacman.move(3)} ?

  \choicec{3} \choicec{0} \choicec{6} \choicec{undefined}

  Que vaut \lstinline{pacman.x} après l'appel \lstinline{f(3)} ?

  \choicec{3} \choicec{0} \choicec{6} \choicec{undefined}
\item \lstset{language=javascript}
\begin{lstlisting}
var obj = {
  x: "b",
  b: 2
};
\end{lstlisting}

  Que vaut \lstinline{obj.x} ?

  \choicec{2} \choicec{"b"} \choicec{obj} \choicec{undefined}

  Que vaut \lstinline{obj["b"]} ?

  \choicec{2} \choicec{"b"} \choicec{obj} \choicec{undefined}

  Que vaut \lstinline{obj[obj.x]} ?

  \choicec{2} \choicec{"b"} \choicec{obj} \choicec{undefined}
\item Une fonction est un objet.

   \choice{Vrai} \choice{Faux}
\item \lstset{language=javascript}
\begin{lstlisting}
var P = {
   x: 2,
   a: "a"
};
var M = {};
\end{lstlisting}

  Que vaut \lstinline{M.x} ?

  \choicec{2} \choicec{3} \choicec{"a"} \choicec{undefined}

  Que vaut \lstinline{P.x} ?

  \choicec{2} \choicec{3} \choicec{"a"} \choicec{undefined}
\item \lstset{language=javascript}
\begin{lstlisting}
var P = {
   x: 2,
   a: "a"
};
var M = {
  __proto__: P,
};
\end{lstlisting}

  Que vaut \lstinline{M.x} ?

  \choicec{2} \choicec{3} \choicec{"a"} \choicec{undefined}

  Que vaut \lstinline{P.x} ?

  \choicec{2} \choicec{3} \choicec{"a"} \choicec{undefined}
\item \lstset{language=javascript}
\begin{lstlisting}
var P = {
   x: 2,
   a: "a"
};
var M = {
  __proto__: P,
  x: 3
};
var J = {
  __proto__: M
};
\end{lstlisting}

  Que vaut \lstinline{M.x} ?

  \choicec{2} \choicec{3} \choicec{"a"} \choicec{undefined}

  Que vaut \lstinline{J.x} ?

  \choicec{2} \choicec{3} \choicec{"a"} \choicec{undefined}

  Que vaut \lstinline{J.a} ?

  \choicec{2} \choicec{3} \choicec{"a"} \choicec{undefined}
\item \lstset{language=javascript}
\begin{lstlisting}
function flag(value) {
  this.value = value;
}

flag.prototype.toggle = function() { this.value = !this.value; }

var t = new flag(true);
t.toggle();
\end{lstlisting}

  L'objet \lstinline{t} a pour prototype l'objet
  \lstinline{flag.prototype}.

  \choice{Vrai} \choice{Faux}

  Que vaut \lstinline{t.value} ?

  \choicec{true} \choicec{false} \choicec{undefined}

  Que vaut \lstinline{flag.value} ?

  \choicec{true} \choicec{false} \choicec{undefined}
\item (Optionnel) Dessinez un mouton.
\end{enumerate}

\end{document}
