% Created 2012-12-11 Tue 07:56
\documentclass[a4paper, 12pt]{article}
\usepackage{fontspec}
\usepackage{luatextra}
\usepackage{hyperref}
\usepackage{graphicx}
\addtolength{\topmargin}{-.8in}
\addtolength{\textheight}{1.6in}
\setmainfont[Mapping=tex-text]{Liberation Serif}
\newfontfamily\progfont[Scale=MatchLowercase]{Ubuntu Mono}
\usepackage{wasysym}
\newcommand{\choice}[1]{\Square\hspace{2pt} #1\hspace{5pt}}
\newcommand{\choicec}[1]{\Square\hspace{2pt} \lstinline{#1}\hspace{5pt}}
\usepackage{listings}
\lstset{basicstyle=\progfont}
\lstset{tabsize=3, columns=fullflexible, keepspaces=true}
\lstset{breaklines=false, showstringspaces=false}
\lstset{frame=leftline, framerule=1pt}
\lstset{framesep=2pt}
\lstdefinelanguage{JavaScript}{
keywords={typeof, new, true, false, catch, function, return, null, catch, switch, var, if, in, while, do, else, case, break},
keywordstyle=\bfseries,
ndkeywords={class, export, boolean, throw, implements, import, this},
ndkeywordstyle=\bfseries,
identifierstyle=,
sensitive=false,
comment=[l]{//},
morecomment=[s]{/*}{*/},
morestring=[b]',
morestring=[b]"
}
\providecommand{\alert}[1]{\textbf{#1}}

\title{Quiz JavaScript}
\author{Languages de programmation II FIL/A2}
\date{7 novembre 2014}

\begin{document}

\maketitle

\textbf{Nom} :

Sans machine ni document.

Durée : 15 minutes.

Cocher la bonne réponse (il peut y en avoir plusieurs).

\begin{enumerate}
\item \lstset{language=javascript}
\begin{lstlisting}
var a;
a = 2 + 3;
\end{lstlisting}

  Que vaut \lstinline{a} ?

  \choicec{5} \choicec{"a"} \choicec{undefined} \choice{Donald Duck}
\item \lstset{language=javascript}
  Que vaut \lstinline{0 == "0"} ?

  \choicec{true} \choicec{false} \choicec{0} \choicec{undefined}

  Que vaut \lstinline{0 === "0"} ?

  \choicec{true} \choicec{false} \choicec{0} \choicec{undefined}
\item \lstset{language=javascript}
\begin{lstlisting}
var f = function (x,y) {
   return y;
}
\end{lstlisting}

   Que vaut \lstinline{f(1,2)} ?

   \choicec{1} \choicec{2} \choicec{undefined}

   Que vaut \lstinline{f(1)} ?

   \choicec{1} \choicec{2} \choicec{undefined}
\item \lstset{language=javascript}
\begin{lstlisting}
function f(x) {
  return function(y) {
    return x * y;
  };
}
\end{lstlisting}

  Que retourne \lstinline{f(3)} ?

  \choicec{3} \choicec{9} \choice{une fonction} \choicec{undefined}

  Que retourne \lstinline{f(3)(3)} ?

  \choicec{3} \choicec{9} \choice{une fonction} \choicec{undefined}
\item \lstset{language=javascript}
\begin{lstlisting}
var obj = {
  x: 12,
  y: function() { return 5 + obj.x; },
};
delete obj.x;
obj.z = obj.x;
obj.x = 6;
\end{lstlisting}

  Que vaut \lstinline{obj.x} ?

  \choicec{12} \choicec{5} \choicec{17} \choicec{6} \choicec{11} \choicec{undefined}

  Que vaut \lstinline{obj.y()} ?

  \choicec{12} \choicec{5} \choicec{17} \choicec{6} \choicec{11} \choicec{undefined}

  Que vaut \lstinline{obj.z} ?

  \choicec{12} \choicec{5} \choicec{17} \choicec{6} \choicec{11} \choicec{undefined}
\item \lstset{language=javascript}
\begin{lstlisting}
var obj = {
  x: "b",
  b: 2
};
\end{lstlisting}

  Que vaut \lstinline{obj.x} ?

  \choicec{2} \choicec{"b"} \choicec{obj} \choicec{undefined}

  Que vaut \lstinline{obj["b"]} ?

  \choicec{2} \choicec{"b"} \choicec{obj} \choicec{undefined}

  Que vaut \lstinline{obj[obj.x]} ?

  \choicec{2} \choicec{"b"} \choicec{obj} \choicec{undefined}
\item Un tableau est aussi un objet.

   \choice{Vrai} \choice{Faux}
\item \lstset{language=javascript}
\begin{lstlisting}
var point = {
  x: 0,
  move: function(dx) {
    this.x += dx;
  }
};
var f = point.move;
point.move(3);
f(3);
\end{lstlisting}

  %% Le mot-clé \lstinline{this} de la fonction \lstinline{move} est lié
  %% à l'objet \lstinline{point} lors de la définition de l'objet.

  %% \choice{Vrai} \choice{Faux}

  Que vaut \lstinline{point.x} ?

  \choicec{0} \choicec{3} \choicec{6} \choicec{undefined}

  Que vaut \lstinline{f.x} ?

  \choicec{0} \choicec{3} \choicec{6} \choicec{undefined}
\item \lstset{language=javascript}
\begin{lstlisting}
var P = {
   x: 2,
   y: "y",
};
var M = {
  __proto__: P,
  y: 3,
};
\end{lstlisting}

  Que vaut \lstinline{M.x} ?

  \choicec{2} \choicec{3} \choicec{"y"} \choicec{undefined}

  Que vaut \lstinline{M.y} ?

  \choicec{2} \choicec{3} \choicec{"y"} \choicec{undefined}
\item \lstset{language=javascript}
\begin{lstlisting}
var P = {
   x: 2,
   y: "y",
};
function M(x) {
  return {
    __proto__: P,
    x: x,
  };
};
var m = M(3);
P.y = 2;
\end{lstlisting}

  Que vaut \lstinline{m.x} ?

  \choicec{2} \choicec{3} \choicec{"y"} \choicec{undefined}

  Que vaut \lstinline{m.y} ?

  \choicec{2} \choicec{3} \choicec{"y"} \choicec{undefined}
\item \lstset{language=javascript}
\begin{lstlisting}
function M(x) {
  this.x = x;
}

M.prototype.toggle = function() { this.x = !this.x; }

var m = new M(true);
m.toggle();
\end{lstlisting}

  L'objet \lstinline{m} a pour prototype l'objet
  \lstinline{M.prototype}.

  \choice{Vrai} \choice{Faux}

  Que vaut \lstinline{m.x} ?

  \choicec{true} \choicec{false} \choicec{undefined}

  Que vaut \lstinline{M.x} ?

  \choicec{true} \choicec{false} \choicec{undefined}
\item Écrivez la fonction \lstinline{map(tableau, f)} qui, pour un
  tableau \lstinline{[a,b,c,...]} renvoie un nouveau tableau
  \lstinline{[f(a),f(b),f(c),...]}.
\end{enumerate}

\end{document}
