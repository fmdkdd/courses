\documentclass[a4paper, 12pt]{article}
\addtolength{\topmargin}{-.8in}
\addtolength{\textheight}{1.6in}

\usepackage{lastpage}
\usepackage{fancyhdr}
\fancypagestyle{plain}{%
  \fancyfoot[C]{\thepage{}/\pageref{LastPage}}
}
\renewcommand{\headrulewidth}{0pt}
\pagestyle{plain}

\usepackage[bitstream-charter]{mathdesign}
\usepackage{fontspec}
%% \setmainfont{Charter}
\setmonofont[Scale=MatchLowercase]{Fira Mono}

\usepackage{wasysym}
\newcommand{\choice}[1]{\Square\hspace{2pt} #1\hspace{5pt}}
\newcommand{\choicec}[1]{\Square\hspace{2pt} \lstinline{#1}\hspace{5pt}}

\usepackage[usenames, dvipsnames]{xcolor}
\newcommand{\fullpoint}[1]{\textcolor{RubineRed}{#1}}
\newcommand{\halfpoint}[1]{\textcolor{SkyBlue}{#1}}
\newcommand{\choiceg}[1]{\fullpoint{\XBox\hspace{2pt} #1\hspace{5pt}}}
\newcommand{\choicegh}[1]{\halfpoint{\XBox\hspace{2pt} #1\hspace{5pt}}}
\newcommand{\choicecg}[1]{\fullpoint{\XBox\hspace{2pt} \lstinline{#1}\hspace{5pt}}}
\newcommand{\choicecgh}[1]{\halfpoint{\XBox\hspace{2pt} \lstinline{#1}\hspace{5pt}}}

\usepackage{listings}
\lstset{basicstyle=\ttfamily}
\lstset{tabsize=3, columns=fullflexible, keepspaces=true}
\lstset{breaklines=false, showstringspaces=false}
\lstset{frame=leftline, framerule=1pt}
\lstset{framesep=2pt}
\lstdefinelanguage{JavaScript}{
keywords={new, true, false, try, catch, var, let, function, return, null,
  undefined, switch, if, in, of, while, do, else, case, break, async, await},
keywordstyle=\bfseries,
ndkeywords={class, export, boolean, throw, implements, import, this},
ndkeywordstyle=\bfseries,
identifierstyle=,
sensitive=false,
comment=[l]{//},
morecomment=[s]{/*}{*/},
morestring=[b]',
morestring=[b]"
}

\providecommand{\alert}[1]{\textbf{#1}}

\title{Quiz JavaScript}
\author{FIL A2}
\date{15 janvier 2019}

\begin{document}

\maketitle

\textbf{Nom} :

Sans machine ni document.

Durée : 25 minutes.

Cocher la bonne réponse (il peut y en avoir plusieurs).

\begin{enumerate}
\item \lstset{language=javascript}
\begin{lstlisting}
 a = 1 + 2 / 4;
\end{lstlisting}

  Que vaut \lstinline{a} ?

  \choicec{6/4} \choicec{3/2} \choicecg{1.5} \choicec{1} \choicec{2}

\item \lstset{language=javascript}
\begin{lstlisting}
 var a;
 var a = 2;
\end{lstlisting}

  Que vaut \lstinline{a} ?

  \choicec{"2"} \choicecg{2} \choicec{"a"} \choicec{undefined} \choice{C'est une erreur}
\item \lstset{language=javascript}
\begin{lstlisting}
 a = 2;
 let a;
\end{lstlisting}

  Que vaut \lstinline{a} ?

  \choicec{"2"} \choicec{2} \choicec{"a"} \choicec{undefined} \choiceg{C'est une erreur}
\item \lstset{language=javascript}
\begin{lstlisting}
 const a = [1,2];
 a.push(3);
\end{lstlisting}

  Que vaut \lstinline{a[2]} ?

  \choicec{1} \choicec{2} \choicecg{3} \choicec{undefined} \choice{C'est une erreur}
\item \lstset{language=javascript}
\begin{lstlisting}
 1 == true;
\end{lstlisting}

  Que vaut cette expression ?

  \choicecg{true} \choicec{false} \choicec{1} \choicec{undefined} \choice{C'est une erreur}

\item \lstset{language=javascript}
\begin{lstlisting}
 1 === true;
\end{lstlisting}

  Que vaut cette expression ?

  \choicec{true} \choicecg{false} \choicec{1} \choicec{undefined} \choice{C'est une erreur}
\item \lstset{language=javascript}
\begin{lstlisting}
 null == undefined;
\end{lstlisting}

  Que vaut cette expression ?

  \choicecg{true} \choicec{false} \choicec{null} \choicec{undefined}
\item \lstset{language=javascript}
\begin{lstlisting}
 'ab2' === ('ab' + 2);
\end{lstlisting}

  Que vaut cette expression ?

  \choicecg{true} \choicec{false} \choicec{'ab2'} \choicec{'abtwo'}
  \choicec{undefined}\\\choice{C'est une erreur}
\item \begin{lstlisting}
 function f(x,y) {
   return y;
 }
\end{lstlisting}

  Que vaut \lstinline{f(1,2)} ?

  \choicec{1} \choicecg{2} \choicec{undefined} \choice{C'est une erreur}

  Que vaut \lstinline{f(1)} ?

  \choicec{1} \choicec{2} \choicecg{undefined} \choice{C'est une erreur}

  Que vaut \lstinline{f(1,2,3)} ?

  \choicec{1} \choicecg{2} \choicec{3} \choicec{undefined} \choice{C'est une erreur}
\item \lstset{language=javascript}
\begin{lstlisting}
 let r = [];
 for (var i = 0; i < 3; ++i) {
   r.push(x => x + i);
 }
 let t = r.map(f => f(3))
\end{lstlisting}

  Que vaut \lstinline{t} ?

  \choicec{[0,1,2]} \choicec{[3,4,5]} \choicec{[3,3,3]} \choicecg{[6,6,6]}

\newpage
\item \lstset{language=javascript}
\begin{lstlisting}
 let r = [];
 for (let i = 0; i < 3; ++i) {
   r.push(i => i + i);
 }
 let t = r.map(f => f(3))
\end{lstlisting}

  Que vaut \lstinline{t} ?

  \choicec{[0,1,2]} \choicec{[3,4,5]} \choicec{[3,3,3]} \choicecg{[6,6,6]}

\item \lstset{language=javascript}
\begin{lstlisting}
 function f(x) {
   let o = [x];
   return function(y) {
     o.push(y);
     return o.length;
   };
 }
 let g = f('b');
 g('c');
 g('d');
\end{lstlisting}

  Que retourne \lstinline{f('a')} ?

  \choicec{0} \choicec{1} \choicec{2} \choicec{3} \choiceg{une fonction} \choicec{undefined}

  Que retourne \lstinline{f('a')('a')} ?

  \choicec{0} \choicec{1} \choicecg{2} \choicec{3} \choice{une fonction} \choicec{undefined}

  Que retourne l'appel \lstinline{g('c')} ?

  \choicec{0} \choicec{1} \choicecg{2} \choicec{3} \choice{une fonction} \choicec{undefined}

  Que retourne l'appel \lstinline{g('d')} ?

  \choicec{0} \choicec{1} \choicec{2} \choicecg{3} \choice{une fonction} \choicec{undefined}
\item \lstset{language=javascript}
\begin{lstlisting}
 let obj = {
   a: "b",
   b: 2
 };
\end{lstlisting}

  Que vaut \lstinline{obj.a} ?

  \choicec{2} \choicec{"a"} \choicecg{"b"} \choicec{obj} \choicec{undefined}

  Que vaut \lstinline{obj[2]} ?

  \choicec{2} \choicec{"a"} \choicec{"b"} \choicec{obj} \choicecg{undefined}

  Que vaut \lstinline{obj[obj.a]} ?

  \choicecg{2} \choicec{"a"} \choicec{"b"} \choicec{obj} \choicec{undefined}
\newpage
\item \begin{lstlisting}
 let P = {
   x: 0,
   y: 1,
 };
 let O = {__proto__: P,
   x: 2,
 };
 O.y = 3;
\end{lstlisting}

  Que vaut \lstinline{O.x} ?

  \choicec{0} \choicec{1} \choicecg{2} \choicec{3} \choicec{undefined}

  Que vaut \lstinline{O.y} \emph{avant} l'affectation \lstinline{O.y = 3} ?

  \choicec{0} \choicecg{1} \choicec{2} \choicec{3} \choicec{undefined}

  Que vaut \lstinline{O.y} \emph{après} l'affectation \lstinline{O.y = 3} ?

  \choicec{0} \choicec{1} \choicec{2} \choicecg{3} \choicec{undefined}

  Que vaut \lstinline{P.y} \emph{après} l'affectation \lstinline{O.y = 3} ?

  \choicec{0} \choicecg{1} \choicec{2} \choicec{3} \choicec{undefined}
\item \begin{lstlisting}
 let O = {
   m(i) {
     this.i = i;
     return function(y) { this.i += y; };
   },
   g(i) {
     this.i = i;
     return y => { this.i += y; };
   }
 };
 let m = O.m(1);
 m(1);

 let g = O.g(1);
 g(1);

 let f = O.m;
 f(1);
\end{lstlisting}

  Que vaut \lstinline{O.i} après l'appel \lstinline{m(1)} ?

  \choicec{0} \choicecg{1} \choicec{2} \choicec{3} \choicec{undefined}

  Que vaut \lstinline{O.i} après l'appel \lstinline{g(1)} ?

  \choicec{0} \choicec{1} \choicecg{2} \choicec{3} \choicec{undefined}

  Que vaut \lstinline{O.i} après l'appel \lstinline{f(1)} ?

  \choicec{0} \choicec{1} \choicecg{2} \choicec{3} \choicec{undefined}
\item \begin{lstlisting}
 let P = { x: 0 };
 let T = { x: 1 };
 let O = {__proto__: P };
 O.__proto__ = T;
\end{lstlisting}

  Que vaut \lstinline{O.x} \emph{avant} l'affectation \lstinline{O.__proto__ = T} ?

  \choicecg{0} \choicec{1} \choicec{undefined}

  Que vaut \lstinline{O.x} \emph{après} l'affectation \lstinline{O.__proto__ = T} ?

  \choicec{0} \choicecg{1} \choicec{undefined}
\item La fonction \lstinline{log(x)} affiche \lstinline{x} dans la console.  La
  fonction \lstinline{sleep(n)} retourne une promesse qui se résout après
  \lstinline{n} secondes.
  \begin{lstlisting}
 async function f() {
   log(1)
   sleep(10).then(log(2))
   log(3)
   await sleep(20).then(log(4))
   log(5)
 }
 f()
\end{lstlisting}

  Après combien de secondes se termine le programme ?  (On suppose que toute
  opération autre que \lstinline{sleep} prend un temps insignifiant.)

  \choice{0} \choice{10} \choiceg{20} \choice{30}

  Donnez l'affichage de la console:

  {\color{RubineRed} Une erreur s'est glissée dans l'énoncé.  Supposons que
    \lstinline{log} et \lstinline{sleep} sont définies ainsi:
    \begin{lstlisting}
 function log(x) { console.log(x) }
 function sleep(n) {
   return new Promise(resolve => setTimeout(resolve, n * 1000))
 }
    \end{lstlisting}

    Tel que l'énoncé est écrit, l'affichage sur la console sera
    \framebox{1 2 3 4 5} car
    \lstinline{log(2)} et \lstinline{log(4)} affichent \emph{directement} sur la
    console plutôt que d'être appelés par \lstinline{then}, puisque
    \lstinline{log(2)} ne renvoit pas une fonction.

    L'enoncé aurait dû être:
    \begin{lstlisting}
 async function f() {
   log(1)
   sleep(10).then(() => log(2))
   log(3)
   await sleep(20).then(() => log(4))
   log(5)
 }
 f()
    \end{lstlisting}
    où l'on passe des fonctions anonymes à \lstinline{then} pour que
    \lstinline{log(2)} et \lstinline{log(4)} soient appelés à la fin du timeout.

    Avec ce changement l'affichage sur la console est \framebox{1 3 2 4 5}.
  }

\item Écrivez la fonction \lstinline{zipWith(a, b, f)} où \lstinline{a} et
  \lstinline{b} sont deux tableaux et \lstinline{f} une fonction de deux
  arguments.  La fonction \lstinline{zipWith} retourne le tableau
  \lstinline{[f(a[0], b[0]), f(a[1], b[1]), ...]}.  Le tableau retourné par
  \lstinline{zipWith} est de longueur égale au plus petit tableau passé en
  arguments.
  \lstset{basicstyle=\ttfamily\color{RubineRed}}
  \begin{lstlisting}
 function zipWith(a, b, f) {
   let r = []
   let l = Math.min(a.length, b.length)
   for (let i = 0; i < l; ++i) {
     r.push(f(a[i], b[i]))
   }
   return r
 }\end{lstlisting}
\end{enumerate}

\end{document}
